

Anforderungen an Visualisierungen anfangen

	Struktur?
	Einleitung, klar...
	dann was gibt es für Anforderungen an Visualisierungen?
	
	erst mal viszualisation pipeline, bild sieht gantz gut aus. Als nächstes in "readings in blabla " lesen ab vor data tables
	schreiben mit inkluden 1.4.2 ausinteractive data visualaisation
	Da gibt es visuelle variablen
	taxonomien


NEXT:

	VISUALISIERUNGEN DIE FEHLEN IN CHAP 2 EINBAUEN
	INDEX CHART WURDE MIT D3 (PROTOVIS) gemacht


	NACH URLAUB:
	KORF FEEEDBACK ANALYSIEREN
	ALGORITHEMN ZUR AUTOMATISCHEN DATEN FILTERUNG: PCA;MDS;SOM bib link in -> Temporal MDS Plots for Analysis of Multivariate Data
	READINGS IN VISUALIZATION AN TUHH AUSLEIHEN
	VOR DATA TABLES STEHT ABLAUF DER VIS PIPELINE NOCHMAL ERKLÄRT

CHAP 4 ->

Hauptsächlich wiedergeben was in den papers so steht




SAVED NOTES (FOR CLEANER DOCUMENT):

(ca.)LINE	CHAPTER		TEXT
	510		Anford.		Wie hilft diese Vis Technik/Werkzeug bei welchem Fehler oder Beobachtung auch immer?

						Motivation für diese Anforderung?

						80:20! welche 20\% Anf. sind für 80\% der Infos verantwortlich??\newline
						->>>> SCALABILITY;STORYTELLING?;EFFECTIV EVAL?

						A systematic review -> 2. Absatz in Conclusion \cite{isenberg_systematic_2013} \newline
						empirical studies in \cite{lam_empirical_2012}

	880		Evalbasis	\TODO{Anforderungen an diese Methoden und was kann man daraus gewinnen}

						Wie Eval andere? -> Kriterien / Nutzen?

						Aus verschiedenen Evaluationsmöglichkeiten ergeben sich verschiedene Anforderungen

						Evaluierungsszenarien und Evaluierungspraktiken

						\TODO{Evaluationstechnikwahl abgrenzen von anderen die entweder das gleiche gemacht haben oder das hat eben so noch keiner angewandt}

